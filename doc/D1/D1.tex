\documentclass[11pt]{article}
\usepackage{acl2014}
\usepackage{times}
\usepackage{url}
\usepackage{latexsym}

% Change this if needed to set titlebox size.
%\setlength\titlebox{5cm}

\title{LING 573: Initial Project Report}

\author{Clara Gordon \\
  University of Washington \\
  Seattle, WA \\
  {\tt cgordon1@uw.edu} \\\And
  Claire Jaja \\
  University of Washington \\
  Seattle, WA \\
  {\tt cjaja@uw.edu} \\\And
  Andrea Kahn \\
  University of Washington \\
  Seattle, WA \\
  {\tt andrea.m.kahn@gmail.com} \\}

\date{}

\begin{document}
\maketitle
\begin{abstract}
    Later, this section will have the abstract - a short high-level overview of the paper, usually 150 words or so.
\end{abstract}

\section{Introduction}

\section{System Overview}

A description of the major design, methodological, and algorithmic decisions in your project. It often includes a schematic of the system architecture.

\section{Approach}

This section should provide the details of the major subcomponents of your system. Your report should include at least the following three major sections correspondign to the project deliverables. You may also include other subsections as appropriate.

\section{Results}

This section should present the major results of the formal evaluation of your system and components.

\section{Discussion}

This section will analyze your results in a bit more detail. It is an appropriate location for error analysis and assessment of the strengths and weaknesses of the different components.

\section{Conclusion}

\nocite{*}
\bibliographystyle{acl}
\bibliography{D1references}

%\begin{thebibliography}{}

%\end{thebibliography}

\end{document}
